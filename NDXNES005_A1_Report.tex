% Options for packages loaded elsewhere
\PassOptionsToPackage{unicode}{hyperref}
\PassOptionsToPackage{hyphens}{url}
%
\documentclass[
  12pt,
]{article}
\usepackage{amsmath,amssymb}
\usepackage{iftex}
\ifPDFTeX
  \usepackage[T1]{fontenc}
  \usepackage[utf8]{inputenc}
  \usepackage{textcomp} % provide euro and other symbols
\else % if luatex or xetex
  \usepackage{unicode-math} % this also loads fontspec
  \defaultfontfeatures{Scale=MatchLowercase}
  \defaultfontfeatures[\rmfamily]{Ligatures=TeX,Scale=1}
\fi
\usepackage{lmodern}
\ifPDFTeX\else
  % xetex/luatex font selection
\fi
% Use upquote if available, for straight quotes in verbatim environments
\IfFileExists{upquote.sty}{\usepackage{upquote}}{}
\IfFileExists{microtype.sty}{% use microtype if available
  \usepackage[]{microtype}
  \UseMicrotypeSet[protrusion]{basicmath} % disable protrusion for tt fonts
}{}
\makeatletter
\@ifundefined{KOMAClassName}{% if non-KOMA class
  \IfFileExists{parskip.sty}{%
    \usepackage{parskip}
  }{% else
    \setlength{\parindent}{0pt}
    \setlength{\parskip}{6pt plus 2pt minus 1pt}}
}{% if KOMA class
  \KOMAoptions{parskip=half}}
\makeatother
\usepackage{xcolor}
\usepackage[margin=1in]{geometry}
\usepackage{graphicx}
\makeatletter
\def\maxwidth{\ifdim\Gin@nat@width>\linewidth\linewidth\else\Gin@nat@width\fi}
\def\maxheight{\ifdim\Gin@nat@height>\textheight\textheight\else\Gin@nat@height\fi}
\makeatother
% Scale images if necessary, so that they will not overflow the page
% margins by default, and it is still possible to overwrite the defaults
% using explicit options in \includegraphics[width, height, ...]{}
\setkeys{Gin}{width=\maxwidth,height=\maxheight,keepaspectratio}
% Set default figure placement to htbp
\makeatletter
\def\fps@figure{htbp}
\makeatother
\setlength{\emergencystretch}{3em} % prevent overfull lines
\providecommand{\tightlist}{%
  \setlength{\itemsep}{0pt}\setlength{\parskip}{0pt}}
\setcounter{secnumdepth}{-\maxdimen} % remove section numbering
% definitions for citeproc citations
\NewDocumentCommand\citeproctext{}{}
\NewDocumentCommand\citeproc{mm}{%
  \begingroup\def\citeproctext{#2}\cite{#1}\endgroup}
\makeatletter
 % allow citations to break across lines
 \let\@cite@ofmt\@firstofone
 % avoid brackets around text for \cite:
 \def\@biblabel#1{}
 \def\@cite#1#2{{#1\if@tempswa , #2\fi}}
\makeatother
\newlength{\cslhangindent}
\setlength{\cslhangindent}{1.5em}
\newlength{\csllabelwidth}
\setlength{\csllabelwidth}{3em}
\newenvironment{CSLReferences}[2] % #1 hanging-indent, #2 entry-spacing
 {\begin{list}{}{%
  \setlength{\itemindent}{0pt}
  \setlength{\leftmargin}{0pt}
  \setlength{\parsep}{0pt}
  % turn on hanging indent if param 1 is 1
  \ifodd #1
   \setlength{\leftmargin}{\cslhangindent}
   \setlength{\itemindent}{-1\cslhangindent}
  \fi
  % set entry spacing
  \setlength{\itemsep}{#2\baselineskip}}}
 {\end{list}}
\usepackage{calc}
\newcommand{\CSLBlock}[1]{\hfill\break\parbox[t]{\linewidth}{\strut\ignorespaces#1\strut}}
\newcommand{\CSLLeftMargin}[1]{\parbox[t]{\csllabelwidth}{\strut#1\strut}}
\newcommand{\CSLRightInline}[1]{\parbox[t]{\linewidth - \csllabelwidth}{\strut#1\strut}}
\newcommand{\CSLIndent}[1]{\hspace{\cslhangindent}#1}
\usepackage{setspace}
\onehalfspacing
\usepackage{etoolbox}
\apptocmd{\thebibliography}{\setlength{\itemsep}{1.0\baselineskip}}{}{}
\usepackage{amsmath}
\usepackage{amssymb}
\usepackage{float}
\usepackage{graphicx}
\usepackage{fvextra}
\usepackage{adjustbox}
\usepackage{tabu}
\usepackage{threeparttable}
\ifLuaTeX
  \usepackage{selnolig}  % disable illegal ligatures
\fi
\usepackage{bookmark}
\IfFileExists{xurl.sty}{\usepackage{xurl}}{} % add URL line breaks if available
\urlstyle{same}
\hypersetup{
  pdftitle={Portfolio Theory: Assignment 1},
  pdfauthor={Nesan Naidoo : NDXNES005},
  hidelinks,
  pdfcreator={LaTeX via pandoc}}

\title{Portfolio Theory: Assignment 1}
\usepackage{etoolbox}
\makeatletter
\providecommand{\subtitle}[1]{% add subtitle to \maketitle
  \apptocmd{\@title}{\par {\large #1 \par}}{}{}
}
\makeatother
\subtitle{The Statistics of Strategy Back-Testing}
\author{Nesan Naidoo : NDXNES005}
\date{2025-09-21}

\begin{document}
\maketitle

\section{Problem Specification}\label{problem-specification}

This assignment investigated the statistics of strategy backtesting.
Part I focused on proving that the estimated annualized Sharpe ratio(SR)
converges asymptotically to a normal distribution. Furthermore, part I
motivated that for a sufficiently large number of samples, the mean of
the sample maximum of standard normally distributed random variables can
be approximated. Lastly, part I focused on the derivation and discussion
of the minimum backtest length.

Part II focused on mean--variance backtesting of the tangency portfolio
under full investment and no-short-selling constraints. Sharpe
ratio--maximising portfolios were calculated from rolling windows of the
historical data. The historical dataset is divided into in-sample (IS)
and out-of-sample (OOS) datasets. Two experiments are conducted: 1)
compared IS and OOS Sharpe ratios and 2) evaluated OOS backtest
performance using a rolling window approach.

\section{Data Specification}\label{data-specification}

The Tactical Asset Allocation data is loaded from
``PT-DATA-ALBI-JIBAR-JSEIND-Daily-1994-2017.xlsx'' with the following
asset information: \textbackslash{} Key: (Number of rows, number of
columns) 1. Sheet 1: (8439,2) ALBI (All Bond Index (ALBI) Total Return
Index (TRI) Data

\begin{enumerate}
\def\labelenumi{\arabic{enumi}.}
\setcounter{enumi}{1}
\item
  Sheet 2: (8405,4) Money Market Data: JIBAR and STEFI TRI
\item
  Sheet 3: (8439,28) JSE ICB Industrial Level Indices
\item
  Sheet 4: (8439,20) JSE Various Indices: JSE Growth, JSE Value, JSE
  ALSI, JSE SRI
\end{enumerate}

\section{Configuration control}\label{configuration-control}

Coding for Part II's Experiment 1 and 2 was completed using RStudio
2024.09.0+375 (``Cranberry Hibiscus'' Release) and was based on R and
MATLAB code provided by Professor Tim Gebbie(STA4028Z).

Version control: managed with Git and GitHub.

To view repository, click on link:
\href{https://github.com/NesanNaidoo/Portfolio-Theory-Assignment-1-Backtesting}{PT-Assignment
1}

\newpage

\section{PART I : Introduction to Strategy
Backtesting}\label{part-i-introduction-to-strategy-backtesting}

\subsection{Question 1 : Sample Error when Estimating the Sharpe
Ratio}\label{question-1-sample-error-when-estimating-the-sharpe-ratio}

Proof below is based on the asymptotic distributions of Sharpe Ratio
estimators (See Appendix A. IID Returns in (Lo, 2002) )

\subsubsection{1.1 Assumptions}\label{assumptions}

\begin{itemize}
\item
  We assume IID excess returns \(r_1,\dots,r_n\) with
  \(r_t \sim \mathcal{N}(\mu,\sigma^2).\)
\item
  Let \(q\) be returns per year, \(y\) the number of years, and
  \(n=qy\).
\item
  The true annualised Sharpe is \(SR=\sqrt{q}\,\frac{\mu}{\sigma}.\)
\item
  Estimators used in proof: \[
  \widehat\mu=\frac{1}{n}\sum_{t=1}^n r_t,\quad
  \widehat\sigma^2=\frac{1}{n}\sum_{t=1}^n (r_t-\widehat\mu)^2,\quad
  \widehat{SR}=\sqrt{q}\,\frac{\widehat\mu}{\widehat\sigma}.
  \]
\end{itemize}

\subsubsection{1.2 Distribution of the sample
mean}\label{distribution-of-the-sample-mean}

Since \(r_t\sim \mathcal{N}(\mu,\sigma^2)\) and using the fact the the
sum of independent Normal variables are Normal. Therefore,\[
\sum_{t=1}^n r_t \sim N\!\big(n\mu,\; n\sigma^2\big). \tag{1}
\]

Therefore, the sample mean \(\widehat\mu=\dfrac{1}{n}\sum_{t=1}^n r_t\)
follows a Normal distribution \[
\widehat\mu \sim N\!\Big(\mu,\; \frac{\sigma^2}{n}\Big). \tag{2}
\]

Hence, by centering and scaling by \(\sqrt{n}\) , this leads to \[
\sqrt{n}\,(\widehat\mu-\mu) \sim \mathcal{N}(0,\sigma^2). \tag{3}
\]

\subsubsection{1.3 Distribution of the sample
variance}\label{distribution-of-the-sample-variance}

Given
\(\widehat\sigma^2 \;=\; \frac{1}{n}\sum_{t=1}^n (r_t - \widehat\mu)^2\)
, then the result based on Normal theory is
\(\frac{n\widehat\sigma^2}{\sigma^2} \;\sim\; \chi^2_{\,n-1}.\)

If \(U \sim \chi^2_k\) then \(\mathbb{E}[U]=k\) and
\(\operatorname{Var}(U)=2k.\)

Hence, to standardise \(U\) we subtract its mean \(k\) and divide by its
standard deviation \(\sqrt{2k}\).\\
Let \(k=n-1\) and \(U = \dfrac{n\widehat\sigma^2}{\sigma^2}\) . Using
the fact that \(\chi_{n-1}^2\), is the sum of \(n-1\) independent
\(Z_i^2\) terms, where each \(Z_i^2\) has mean 1 and variance 2 and by
the Central Limit Theorem , the centered and scaled sum converges to
\(\mathcal{N}(0,1)\) as \(n \to \infty\). Therefore,

\[
\frac{\dfrac{n\widehat\sigma^2}{\sigma^2}-(n-1)}{\sqrt{2(n-1)}} \;\xrightarrow{d}\; \mathcal{N}(0,1), \qquad (n\to\infty)   \tag{4}.
\]

Next the numerator and denominator is multiplied by \(\sigma^2\): \[
\frac{n\widehat\sigma^2-(n-1)\sigma^2}{\sigma^2\sqrt{2(n-1)}} \xrightarrow{d} \mathcal{N}(0,1) \tag{5}.
\]

Then the numerator can be rewritten as
\(\mathcal{N}(\widehat\sigma^2-\sigma^2)+\sigma^2\) and fraction can be
split into 2 terms: \[
\frac{\mathcal{N}(\widehat\sigma^2-\sigma^2)}{\sigma^2\sqrt{2(n-1)}} + \frac{1}{\sqrt{2(n-1)}} \xrightarrow{d} \mathcal{N}(0,1) \tag{6}.
\] Since \(\dfrac{1}{\sqrt{2(n-1)}}\to 0\) as \(n\to\infty\), therefore
\[
\frac{\mathcal{N}(\widehat\sigma^2-\sigma^2)}{\sigma^2\sqrt{2(n-1)}} \xrightarrow{d} \mathcal{N}(0,1) \tag{7}.
\]

Let \(A_n=\sqrt{n}(\widehat\sigma^2-\sigma^2)\). Therefore, \[
\frac{\sqrt{n}\,A_n}{\sigma^2\sqrt{2(n-1)}} \xrightarrow{d} \mathcal{N}(0,1) \tag{8}.
\] Hence
\(A_n \xrightarrow{d} N\!\left(0,\frac{\sigma^4 2(n-1)}{n}\right)\), and
letting \(n\to\infty\) leads to \[
\sqrt{n}(\widehat\sigma^2-\sigma^2)\xrightarrow{d} \mathcal{N}(0,2\sigma^4) \tag{9}.
\]

\subsubsection{\texorpdfstring{1.4 Using Delta method to obtain
\(\sqrt{n}(\widehat\sigma-\sigma)\)}{1.4 Using Delta method to obtain \textbackslash sqrt\{n\}(\textbackslash widehat\textbackslash sigma-\textbackslash sigma)}}\label{using-delta-method-to-obtain-sqrtnwidehatsigma-sigma}

Let \(h(x)=\sqrt{x}\), so \(h'(\sigma^2)=1/(2\sigma)\). Then by Taylor
expansion, \[ 
\widehat\sigma-\sigma \approx h'(\sigma^2),(\widehat\sigma^2-\sigma^2)=1/(2\sigma)(\widehat\sigma^2-\sigma^2) \tag{10}.
\] By multiplying by \(\sqrt{n}\) : \[ 
\sqrt{n}(\widehat\sigma-\sigma) \approx \frac{\sqrt{n}}{2\sigma}(\widehat\sigma^2-\sigma^2) \tag{11} 
\]

Using (9): \[
\sqrt{n}(\widehat\sigma-\sigma)\xrightarrow{d} N\!\Big(0,\Big(\frac{1}{2\sigma}\Big)^2\cdot 2\sigma^4\Big)=N\!\Big(0,\frac{\sigma^2}{2}\Big) \tag{12}.
\]

\subsubsection{1.5 Combining results so
far}\label{combining-results-so-far}

\[
\sqrt{n}\begin{pmatrix}\widehat\mu-\mu\\[4pt]\widehat\sigma-\sigma\end{pmatrix}
\xrightarrow{d} N\!\Big(0,\ \Sigma\Big)  \qquad\text{where}\qquad
\Sigma=\begin{pmatrix}\sigma^2 & 0\\[6pt]0 & \sigma^2/2\end{pmatrix}.
\]

\subsubsection{1.6 Using multivariate delta
method}\label{using-multivariate-delta-method}

Let \(g(\mu,\sigma)=\sqrt{q}\,\mu/\sigma = SR\) . Then the Taylor
expansion leads to \[
\sqrt{n}\big(g(\widehat\mu,\widehat\sigma)-g(\mu,\sigma)\big)
= \nabla g(\mu,\sigma)^\top \sqrt{n}\begin{pmatrix}\widehat\mu-\mu\\[4pt]\widehat\sigma-\sigma\end{pmatrix} + o_p(1) \tag{13}.
\] Thus by the Central Limit Theorem, \[
\sqrt{n}(\widehat{SR}-SR)\xrightarrow{d} N\!\Big(0,\nabla g^\top\Sigma\nabla g\Big) \tag{14}.
\]

\subsubsection{1.7 Calculating the gradient and
variance}\label{calculating-the-gradient-and-variance}

Partial derivatives: \[
\frac{\partial g}{\partial\mu}=\frac{\sqrt{q}}{\sigma} \qquad \text{and} \qquad
\frac{\partial g}{\partial\sigma}=-\frac{SR}{\sigma},
\] so \(\nabla g=(\sqrt{q}/\sigma,\ -SR/\sigma)^\top\). Then \[
V=\nabla g^\top\Sigma\nabla g
= \Big(\frac{\sqrt{q}}{\sigma}\Big)^2\sigma^2 + \Big(\frac{SR}{\sigma}\Big)^2\frac{\sigma^2}{2}
= q + \frac{SR^2}{2} \tag{15}.
\] Hence, \[
\sqrt{n}(\widehat{SR}-SR)\xrightarrow{d} N\!\Big(0,\; q+\frac{SR^2}{2}\Big).
\]

\subsubsection{1.8 Converting to variance per
year}\label{converting-to-variance-per-year}

Therefore,\[
\operatorname{Var}(\widehat{SR})\approx\frac{q+\tfrac{SR^2}{2}}{n} \tag{16}.
\] By dividing numerator and denominator by \(q\): \[
\frac{q+\tfrac{SR^2}{2}}{n}=\frac{1+\tfrac{SR^2}{2q}}{\,n/q\,}= \frac{1+\tfrac{SR^2}{2q}}{y} \tag{17}.
\] Since \(n/q=y\). Thus the final result is \[
\widehat{SR}\xrightarrow{d} N\!\Big(SR,\ \frac{1+\tfrac{SR^2}{2q}}{y}\Big) \tag{18}.
\]

\newpage

\subsection{Question 2 : The Maximum of the
Sample}\label{question-2-the-maximum-of-the-sample}

The proof below is based on Appendix: Proof of proposition 1 in (Bailey
\& López de Prado, 2014). Foundations for this proof and normalisation
constants \((a_N,b_N)\) were based on (Embrechts et al., 1997)
(pp.138--147) and Proposition 2.1(iii) in (Resnick, 2008).

\subsubsection{Definitions}\label{definitions}

\begin{itemize}
\item
  Let \(X_1,\dots,X_N\) be IID \(\mathcal{N}(0,1)\).
\item
  Define the sample maximum \(M_N := \max_{1\le n\le N} X_n.\)
\item
  Let \(Z(x)=\Phi(x)\) be the standard normal CDF,
\item
  \(\varphi(x)\) to be its PDF, and \(Z^{-1}\) the inverse CDF .
\item
  Let \(\gamma= 0.5772156649...\) denote the Euler--Mascheroni constant.
\end{itemize}

\subsubsection{2.1 Fisher--Tippett--Gnedenko theorem (Fisher \& Tippett,
1928)}\label{fishertippettgnedenko-theorem-fisher1928}

The Fisher--Tippett--Gnedenko theorem states that for IID samples there
exists normalising sequences \(a_N>0\) and \(b_N\in\mathbb{R}\) such
that the normalised maximum converges in distribution to a
non-degenerate limit which is one of three types (Gumbel, Fréchet,
Weibull). This is determined by the tail of the underlying distribution.

Because the normal distribution has exponentially decaying tails, it
lies in the Gumbel domain . Hence there exist sequences \(a_N>0\) and
\(b_N\in\mathbb{R}\) such that \[
\frac{M_N - b_N}{a_N} \xrightarrow{d} G \tag{1},
\] where \(G(x)=\exp(-e^{-x})\) is the standard Gumbel CDF.

\subsubsection{\texorpdfstring{2.2 Normalising constant (centering):
\(b_N\)}{2.2 Normalising constant (centering): b\_N}}\label{normalising-constant-centering-b_n}

A standard centering is the \((1-1/N)\) quantile.

Define \[
b_N := Z^{-1}\!\Big(1-\frac{1}{N}\Big) \tag{2}.
\] Therefore, by construction\[
1 - Z(b_N) = \frac{1}{N} \tag{3}.
\] This suggests in \(N\) samples, the maximum generally lies near this
quantile.

\subsubsection{\texorpdfstring{2.3 Normalising constant (scaling factor)
:
\(a_N\)}{2.3 Normalising constant (scaling factor) : a\_N}}\label{normalising-constant-scaling-factor-a_n}

Define \(a_N\) to the difference between the \((1-1/(Ne))\) and the
\((1-1/N)\) quantiles. Therefore,\[
a_N := Z^{-1}\!\Big(1-\frac{1}{Ne}\Big) - Z^{-1}\!\Big(1-\frac{1}{N}\Big) \tag{4}.
\] This choice enforces \[
1 - Z(b_N + a_N) = \frac{1}{Ne}.
\] Hence, for a fixed \(x\) such that \(x \in \mathbb{R}\) ,the tail at
\(b_N + a_N x\) will be approximately \(\dfrac{1}{N}e^{-x}\).

This leads to the Gumbel limit: \[
Z(b_N + a_N x)^N \longrightarrow \exp(-e^{-x}) \tag{5}.
\]

\subsubsection{2.4 Convergence of the normalized maximum and
expectation}\label{convergence-of-the-normalized-maximum-and-expectation}

From 2.3 and the Fisher--Tippett--Gnedenko theorem we obtain \[
\frac{M_N - b_N}{a_N} \xrightarrow{d} G.
\]Results for the Gumbel domain (see (Resnick, 2008) Prop. 2.1(iii))
imply convergence of expectations under the integrability conditions
satisfied here: \[
\lim_{N\to\infty} \mathbb{E}\!\Big[\frac{M_N - b_N}{a_N}\Big] = \mathbb{E}[G]=\gamma \tag{6}.
\] Hence for large \(N\), \[
\mathbb{E}[M_N] \approx b_N + \gamma\,a_N \tag{7}.
\]

\subsubsection{2.5 Final substitution}\label{final-substitution}

Substituting the definitions of \(b_N\) and \(a_N\) into
\(\mathbb{E}[M_N]\approx b_N+\gamma a_N\) results in the following
approximation: \[
\mathbb{E}[M_N] \approx Z^{-1}\!\Big(1-\frac{1}{N}\Big)
\;+\;\gamma\Big(Z^{-1}\!\Big(1-\frac{1}{Ne}\Big)-Z^{-1}\!\Big(1-\frac{1}{N}\Big)\Big) \tag{8}.
\]

Distributing \(\gamma\) over the parentheses,

\[
\mathbb{E}[M_N] \approx Z^{-1}\!\Big(1-\frac{1}{N}\Big)
\;+\;\gamma\,Z^{-1}\!\Big(1-\frac{1}{Ne}\Big)
\;-\;\gamma\,Z^{-1}\!\Big(1-\frac{1}{N}\Big) \tag{9}.
\]

Collecting the two terms that multiply
\(Z^{-1}\big(1-\tfrac{1}{N}\big)\) leads to the final result,

\[
\mathbb{E}[M_N] \approx \big[1 - \gamma\big]\,Z^{-1}\!\Big(1-\frac{1}{N}\Big)
\;+\;\gamma\,Z^{-1}\!\Big(1-\frac{1}{Ne}\Big) \tag{10},
\]

for some constant \(\gamma\) (Euler--Mascheroni constant)

\newpage

\subsection{Question 3 : Minimum Backtest
Length}\label{question-3-minimum-backtest-length}

The proof below is based on Theorem 2 in (Bailey \& López de Prado,
2014). Furthermore the proof uses results from the asymptotic
distribution of Sharpe ratio estimators (see Question 1) and the theory
of Normal maxima (see Question 2).

\subsubsection{3.1 Assumptions and initial
setup}\label{assumptions-and-initial-setup}

Using the result from Question 1 (section 1.8), the asymptotic
distribution of the estimated annualised Sharpe ratio is \[
\widehat{SR}\xrightarrow{d} \mathcal{N}\!\Big(SR,\; \frac{1+\tfrac{SR^2}{2q}}{y}\Big),
\] where \(q\) is returns per year, \(y\) the number of years and
\(n=qy\), and \(SR\) denotes the true annualised Sharpe ratio.

As we are testing \(N\) independent strategies, all with true Sharpe
ratio\(SR = 0\) (out-of-sample Sharpe ratio= 0). Therefore, we
substitute \(SR=0\) into the mean and the variance:

\begin{itemize}
\tightlist
\item
  Mean: the asymptotic mean is \(SR\). Hence, \(SR=0\) so mean becomes
  \(0\).
\item
  Variance: \[
  \frac{1+\tfrac{SR^2}{2q}}{y}
  \;\longrightarrow\;
  \frac{1+\tfrac{0^2}{2q}}{y}
  \;=\; \frac{1}{y}.
  \]
\end{itemize}

Therefore, when the true Sharpe ratio is zero, \[
\;\widehat{SR}\xrightarrow{d} \mathcal{N}\!\Big(0,\; \frac{1}{y}\Big)\ \tag{1}; .
\]

We can also write this as,\[
\widehat{SR} \;=\; \frac{1}{\sqrt{y}}\,Z,\qquad\text{where} \qquad  Z\sim\mathcal{N}(0,1) \tag{2},
\] which will be used to derive the minimum backtest length.

\subsubsection{\texorpdfstring{3.2 Maximum across \(N\)
strategies}{3.2 Maximum across N strategies}}\label{maximum-across-n-strategies}

Suppose \(N\) independent strategies are tested.\\
Let \(Z_1,\dots,Z_N \sim \mathcal{N}(0,1)\) be iid, and we can define
the sample maximum

\[
M_N := \max_{1 \le i \le N} Z_i.
\]

Then the maximum observed Sharpe ratio is

\[
\widehat{SR}_{(N)} = \max_{i} \widehat{SR}_i = \frac{1}{\sqrt{y}}\,M_N. \tag{3}
\]

\subsubsection{3.3 Minimum backtest
length}\label{minimum-backtest-length}

Taking expectations in (1):

\[
\mathbb{E}[\widehat{SR}_{(N)}] = \frac{1}{\sqrt{y}}\,\mathbb{E}[M_N]. \tag{4}
\]

Even if the true Sharpe ratio is zero, selecting the strategy with the
highest in-sample Sharpe ratio introduces a positve bias which means
\(\mathbb{E}[\widehat{SR}_{(N)}] > 0\). Therefore, (2) can be rearranged
to obtain an expression for \(y\) in terms of \(\mathbb{E}[M_N]\): \[
y = \frac{(\mathbb{E}[M_N])^2}{\big(\mathbb{E}[\widehat{SR}_{(N)}]\big)^2}. \tag{5}
\]

Here, \(y\) is the backtest length. So by definition, the minimum
backtest length \(T_{\min}\) is the value of \(y\) needed to ensure that
the expected observed maximum Sharpe ratio
\(\mathbb{E}[\widehat{SR}_{(N)}]\) does not exceed a target .

Hence we can write:

\[
T_{\min} = y = \frac{(\mathbb{E}[M_N])^2}{(\mathbb{E}[\widehat{SR}_{(N)}])^2} \tag{6}.
\]

To control for the bias mentioned earlier, the minimum backtest length
\(y\) needs to be long enough so that the expected maximum observed
Sharpe ratio is no greater than a chosen threshold.

From Question 2, we know that \(\mathbb{E}[M_N]\) grows with \(N\), and
we derived an explicit approximation: \[
\mathbb{E}[M_N] \;\approx\; \big[1 - \gamma\big]\,Z^{-1}\!\Big(1-\tfrac{1}{N}\Big)
+ \gamma\,Z^{-1}\!\Big(1-\tfrac{1}{Ne}\Big). \tag{7}
\]

Also from Question 2, the expected maximum of \(N\) i.i.d. standard
Normals grows logarithmically for large \(N\). Therefore, \[
\mathbb{E}[M_N]^2 \;\lesssim\; 2\ln N. \tag{8}
\]

\subsubsection{3.4 Final result}\label{final-result}

Based on (6) and (8), we can deduce the bound:\[
T_{\min} \;<\; \frac{2\ln(N)}{(\,\mathbb{E}[\mathbb{E}[\widehat{SR}_{(N)}]]\,)^2}=\frac{2\ln(N)}{(\,\mathbb{E}[\widehat{SR}_{(N)}]\,)^2}. \tag{9}
\]

where the denominator \((\mathbb{E}[\mathbb{E}[\widehat{SR}_{(N)}]])^2\)
becomes \((\mathbb{ E}[\widehat{SR}_{(N)}])^2\) since the expectation of
a constant is the constant.

Finally, (6) gives the minimum number of years needed to avoid
overfitting from selecting the maximum in-sample Sharpe ratio among
\(N\) strategies.

\subsubsection{3.5 Discussion}\label{discussion}

As the number of strategies \(N\) increases, the minimum backtest length
grows logarithmically with \(N\). This shows that the minimum backtest
length grows as the analyst tries more independent configurations of the
model so as to keep the Sharpe Ratio at a given level.

\newpage

\phantomsection\label{refs}
\begin{CSLReferences}{1}{0}
\bibitem[\citeproctext]{ref-bailey_2014}
Bailey, D. H., \& López de Prado, M. (2014). The deflated sharpe
ratio:correcting for selection bias, BacktestOverfitting, and
non-normality. \emph{The Journal of Portfolio Management}, \emph{40},
94--107. \url{https://doi.org/10.3905/jpm.2014.40.5.094}

\bibitem[\citeproctext]{ref-embrechts_1997}
Embrechts, P., Klüppelberg, C., \& Mikosch, T. (1997). \emph{Modelling
extremal events for insurance and finance}. Springer.

\bibitem[\citeproctext]{ref-fisher1928}
Fisher, R. A., \& Tippett, L. H. C. (1928). \emph{Limiting forms of the
frequency distribution of the largest or smallest member of a sample}
(Vol. 24, pp. 180--190). Cambridge University Press.

\bibitem[\citeproctext]{ref-lo_2002}
Lo, A. W. (2002). The statistics of sharpe ratios. \emph{Financial
Analysts Journal}, \emph{58}, 36--52.
\url{https://doi.org/10.2469/faj.v58.n4.2453}

\bibitem[\citeproctext]{ref-resnick_2008}
Resnick, S. I. (2008). \emph{Extreme values, regular variation and point
processes}. Springer, Cop.

\end{CSLReferences}

\end{document}
